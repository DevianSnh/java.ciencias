\documentclass[a4paper]{article}

\usepackage{anysize}
%márgenes de 1cm para la lectura.
%Configurar a gusto para la impresión.
\marginsize{1cm}{1cm}{0cm}{1cm} 

\usepackage[spanish]{babel}
\usepackage[utf8]{inputenc}
\usepackage{amsmath}
\usepackage{graphicx}
\usepackage[colorinlistoftodos]{todonotes}
\usepackage{hyperref}

\usepackage{color}

\definecolor{pblue}{rgb}{0.13,0.13,1}
\definecolor{pgreen}{rgb}{0,0.5,0}
\definecolor{pred}{rgb}{0.9,0,0}
\definecolor{pgrey}{rgb}{0.46,0.45,0.48}
\definecolor{gray97}{gray}{.97}
\definecolor{gray75}{gray}{.75}
\definecolor{gray45}{gray}{.45}

\usepackage{listings}
\lstset{language=Java,
frame=Ltb,
framerule=0pt,
aboveskip=0.5cm,
framextopmargin=3pt,
framexbottommargin=3pt,
framexleftmargin=0.4cm,
framesep=0pt,
rulesep=.4pt,
backgroundcolor=\color{gray97},
rulesepcolor=\color{black},
%
showstringspaces = false,
basicstyle=\small\ttfamily,
commentstyle=\color{gray45},
keywordstyle=\bfseries,
%
%
numbers=left,
numbersep=15pt,
numberstyle=\tiny,
numberfirstline = false,
breaklines=true,
%
  showspaces=false,
  showtabs=false,
  breakatwhitespace=true,
%  commentstyle=\color{pgreen},
  keywordstyle=\color{pblue},
  stringstyle=\color{pred},
  moredelim=[il][\textcolor{pgrey}]{$$},
  moredelim=[is][\textcolor{pgrey}]{\%\%}{\%\%},
%
    extendedchars=true,
    literate={á}{{\'a}}1 {ã}{{\~a}}1 {é}{{\'e}}1 {í}{{\'i}}1 {ó}{{\'o}}1 {ú}{{\'u}}1,
}
% minimizar fragmentado de listados
\lstnewenvironment{listing}[1][]
{\lstset{#1}\pagebreak[0]}{\pagebreak[0]}


\title{Computación I - Práctica II}

\author{
  Navarro Miranda, Mauricio\\
  \texttt{mauricio@navarromiranda.mx}
  \and
  Arévalo Loyola, Alma Rosario\\
  \texttt{olaverax@gmail.com}
}

\date{\today}

\begin{document}
\maketitle

\begin{abstract}
Comprender el uso de variables y utilizarlas junto con instrucciones \textit{if} para tomar decisiones.
\end{abstract}

\section{Realiza las siguientes conversiones:}
   
\begin{description}
  \item[a)] $1000_8$ a binario. \\
  
		En el sistema octal y en el binario, cada dígito indica el número de veces que se suma cada potencia de 8 o de 2, respectivamente, según el orden. 
        $$1000_8 = 1(8^3) + 0(8^2) + 0(8^1) + 0(8^0)$$
        $$= 1(2^9)+0(2^8) +0(2^7) +0(2^6)+ 0(2^5)+ 0(2^4)+0(2^3) + 0(2^2)+0(2^1) +0(2^0)$$
        $$= 1000000000_2$$
  \item[b)] $101110101_2$ a hexadecimal. \\
  
  Como $16=2^4$ podemos agrupar de 4 en 4 dígitos, y cada uno de estos grupos representa un número menor que 16, por lo tanto se puede escribir como un dígito en sistema hexadecimal.
  $$1/0111/0101$$
  $$1_2 =1, 0111_2=7, 0101_2=5$$
  
  De modo que $$101110101_2=175_{16}$$
  
  
  
  \item[c)] $100101100_2$ a decimal. \\
   
   Nuevamente, sumando las potencias de $2$, según el orden de los dígitos. $$100101100_2=1(2^8)+0(2^7)+0(2^6)+1(2^5)+0(2^4)+1(2^3)+1(2^2)+0(2^1)+0(2^0)$$    
$$= 256+32+8+4=300_{10}$$


\item[d)] $125_{10}$ a binario. \\
$125$ expresado como suma de potencias de $2$ es: $$125=64+32+16+8+4+1$$
$$=1(2^6)+1(2^5)+1(2^4)+1(2^3)+1(2^2)+0(2^1)+1(2^0)$$
$$=1111101_2$$
    
    




\end{description}


\section{Indica qué número entero representan los siguientes números binarios, considerando a) entero no signado y b) entero signado:}

Considerando como entero no signado es el mismo procedimiento que en el inciso c) del ejercicio anterior:

\begin{description}
  \item[a.a)] 
  	$1100_2= 1(2^3)+1(2^2)+0(2^1)+0(2^0)= 8+4=12$
  \item[a.b)] 
  $1101_2= 1(2^3)+1(2^2)+0(2^1)+1(2^0)= 8+4+1=13$
  \item[a.c)] 
 $0011_2= 0(2^3)+0(2^2)+1(2^1)+1(2^0)= 2+1=3$

\end{description}

 Considerando como entero signado, el primer dígito representa el signo ('0' significa positivo y '1' negativo).

\begin{description}
\item[b.a)] $1100$
 Quitamos el primer $1$, que significa que el número es negativo, y nos queda $100$, obtenemos su complemento a dos, que es $100$, y convertido a decimal es $$100_2=1(2^2)+0(2^1)+0(2^0) =4$$
 Por lo tanto, éste número  representa al $-4$


\item[b.b)] $1101$
Quitamos el primer $1$, que significa que el número es negativo, y nos queda $101$, obtenemos su complemento a dos, que es $011$, y convertido a decimal es $$11_2= 1(2^1)+1(2^0) =2+1=3$$
 Por lo tanto, éste número  representa al $-3$


\item[b.c)] $0011$
Como su primer dígito es $0$, este número es positivo y se convierte de la manera convencional.
$$11_2= 1(2^1)+1(2^0) =2+1=3$$



\end{description}

\todo[inline, color=green!40]{Los códigos de las aplicaciones 3 a 5 están en las siguientes páginas (un código por página).\\
También están incluidos en el directorio src\\
Y pueden ser consultados en github:\\
\url{https://github.com/mautematico/java.ciencias/tree/master/P_0011}}


\newpage
\section{Escribe una aplicación que lea cinco enteros y que determine e imprima los enteros mayor y menor en el grupo.}
\todo[inline, color=green!40]{src/MinimoYmaximo.java}
\begin{lstlisting}[language=JAVA]
/*
    v:0.1.1
    Programa que lee cinco enteros,
        determina el mínimo y el máximo e imprime a pantalla mínimo y máximo.
    Mauricio Navarro Miranda,Alma Arévalo Loyola
    Facultad de Ciencias
*/
//Importamos el paquete necesario para usar Scanner
import java.util.Scanner;
public class MinimoYmaximo{
    public static void main(String args[]){
        //Preparamos una instancia de Scanner para leer de la entrada estándar
        Scanner entrada = new Scanner(System.in);

        //Pedimos al usuario el primer entero; lo guardamos en actual. Notemos que éste es, hasta ahora, el máximo el mínimo.
        System.out.print("Dame el primer entero... ");
        int actual = entrada.nextInt();
        int maximo = actual;
        int minimo = actual;

        System.out.print("Dame el segundo entero... ");
        //leemos el segundo entero
        actual = entrada.nextInt();
        //Si el entero que acabamos de leer es mayor que nuestro mayor almacenado, actualizamos mayor.
        if(actual > maximo)
            maximo = actual;
        //Análogamente para el menor.
        if(actual < minimo)
            minimo = actual;
    
        //Repetimos tres veces más. Omitiré los comentarios a propósito.
        System.out.print("Dame el tercer entero... ");
        actual = entrada.nextInt();
        if(actual > maximo)
            maximo = actual;
        if(actual < minimo)
            minimo = actual;
 
        System.out.print("Dame el cuarto entero... ");
        actual = entrada.nextInt();
        if(actual > maximo)
            maximo = actual;
        if(actual < minimo)
            minimo = actual;
   
        System.out.print("Dame el quinto entero... ");
        actual = entrada.nextInt();
        if(actual > maximo)
            maximo = actual;
        if(actual < minimo)
            minimo = actual;
    
        //Ya hemos leído los cinco enteros. Ahora, int maximo tiene al máximo de ellos, mientras que el mínimo está en int minimo.
        System.out.printf("El máximo de los enteros es %d, y el mínimo es %d\n", maximo, minimo);
    }
}
\end{lstlisting}


\newpage
\section{Escribe una aplicación que lea un entero y que determine e imprima si es impar o par.}
\todo[inline, color=green!40]{src/ParOno.java}
\begin{lstlisting}[language=JAVA]
/*
    v:0.1
    Programa que lee un entero,    determina e imprime si es par o impar.
    Mauricio Navarro Miranda, Alma Arévalo Loyola
    Facultad de Ciencias.
*/
//Importamos el paquete necesario para usar Scanner
import java.util.Scanner;
public class ParOno{
    public static void main(String args[]){
        //Preparamos una entrada de Scanner, para leer la entrada estándar.
        Scanner entrada = new Scanner(System.in);

        //Pedimos al usuario que nos proporcione el entero que vamos a evaluar.
        System.out.print("Dame un número entero... ");
        //Leemos el siguiente entero en la entrada
        int numero = entrada.nextInt();

        //Si el número es congruente a 1 (módulo 2), entonces es impar.
        if(numero % 2 == 1)
            System.out.printf("%d es un número impar\n", numero);
        //Si el número es congruente a 0 (módulo 2), entonces es par.
        if(numero % 2 == 0)
            System.out.printf("%d es un número par\n", numero);

    }
}
\end{lstlisting}

\newpage
\section{Escribe una aplicación que calcule los cuadrados y cubos de los números del 0 al 10, y que imprima los valores resultantes en formato de tabla.}
\todo[inline, color=green!40]{src/Potencias.java}
\begin{lstlisting}[language=JAVA]
/*
    v:0.1
    Programa que calcula e imprime los cuadrados y los cubos de los primeros once naturales.
    Mauricio Navarro Miranda, Alma Arévalo Loyola
    Facultad de Ciencias.
*/
public class Potencias{
    public static void main(String args[]){
        int numero = 0;

        System.out.println("Número\tCuadrado\tCubo");
        //Iniciamos con numero = 0
        System.out.printf("%d\t%d\t\t%d\n", numero, numero*numero, numero*numero*numero);
        
        /*
            Ahora, haremos diez veces lo siguiente:
                incrementaremos el número en 1.
                imprimiremos la fila de la tabla con el nuevo número, su cuadrado y su cubo.
        */
        numero+=1;
        System.out.printf("%d\t%d\t\t%d\n", numero, numero*numero, numero*numero*numero);
        numero+=1;
        System.out.printf("%d\t%d\t\t%d\n", numero, numero*numero, numero*numero*numero);
        numero+=1;
        System.out.printf("%d\t%d\t\t%d\n", numero, numero*numero, numero*numero*numero);
        numero+=1;
        System.out.printf("%d\t%d\t\t%d\n", numero, numero*numero, numero*numero*numero);
        numero+=1;
        System.out.printf("%d\t%d\t\t%d\n", numero, numero*numero, numero*numero*numero);
        numero+=1;
        System.out.printf("%d\t%d\t\t%d\n", numero, numero*numero, numero*numero*numero);
        numero+=1;
        System.out.printf("%d\t%d\t\t%d\n", numero, numero*numero, numero*numero*numero);
        numero+=1;
        System.out.printf("%d\t%d\t\t%d\n", numero, numero*numero, numero*numero*numero);
        numero+=1;
        System.out.printf("%d\t%d\t\t%d\n", numero, numero*numero, numero*numero*numero);
        numero+=1;
        System.out.printf("%d\t%d\t\t%d\n", numero, numero*numero, numero*numero*numero);
    }
}
\end{lstlisting}
\end{document}
